\documentclass[11pt,a4paper,titlepage]{article}
\usepackage[a4paper]{geometry}
\usepackage[utf8]{inputenc}
\usepackage[english]{babel}
\usepackage{lipsum}

\usepackage{amsmath, amssymb, amsfonts, amsthm, fouriernc, mathtools}
% mathtools for: Aboxed (put box on last equation in align envirenment)
% \usepackage{microtype} %improves the spacing between words and letters

\usepackage{graphicx}
\graphicspath{ {./pics/} {./eps/}}
\usepackage{epsfig}
\usepackage{epstopdf}

\usepackage{listings}
\usepackage{verbatim}

\title{WebAssembly with Proof-carrying Code}
\author{Nicolai Nebel Jørgensen - dqz439}
\date{\today}

\begin{document}
\maketitle
\newpage

\section*{Project description}
WebAssembly is a new, binary instruction format for a virtual machine, intended
to be used on the web for both client and server applications. Because it is a binary format
resembling normal assembly, it is well suited as a compiler target for other languages.
WebAssembly is also interesting in a theoretical context: It already has a well-specified formal
semantics, making it well-suited for reasoning about in a mechanical manner.

Since WebAssembly does not, in its core, specify any web-specific behaviour it is not limited to
just being supported inside of web browsers. Nebulet\cite{nebulet} is a Google Summer of Code
project for creating a simulated WebAssembly usermode running in ring 0 in the Linux kernel.\\

Proof-carrying code is a technique for allowing safe execution of untrusted code. A piece
of proof-carrying code consists of two parts: The program itself and a \text{proof} of its safety.
Safety in this context means that the program obeys a set of safety rules defined as a prior.
Then, a user who wants to run a program will first verify its safety according to the safety rules
using a trusted, mechanical \textit{proof validator}. When the program is proof-checked, the untrusted
program can now be considered safe to execute.

In particular, proof-checking a program prior to execution allows greater freedom when executing it safely.
For instance, if a program can be proved to only access memory within a certain range it is not also necessary
to check that this happens at \textit{runtime}. Using this technique, untrusted code can be executed more efficiently
than otherwise safe.\\

I wish to explore applications of proof-carrying code in relation to WebAssembly. To name a couple, this project
will involve exploring the possible gains of proof-carrying techniques,
the formulation of useful security policies code as well as the implementation of said techniques.  
I wish to evaluate the usefulness and applicability of proof-carrying code techniques to WebAssembly in different
domains. This assessment will be informed both by theoretic exploration and practical implementation of the
technique.

% - WASM
    % - What problem does WASM solve
    % - Why has it been implemented
    % - Compilation target
    % - Projects for running WASM in kernel ring 0
% - PCC
    % - What is it, what problem does it solve
    % - Disabling run-time checks with PCC
% - Exploring applications of PCC with WASM

\section*{Learning Objectives}
\begin{enumerate}
    \item
        Explain the principles of Proof-carrying code techniques.
    \item
        Explain the semantics of WebAssembly and the structure of its virtual machine.
    \item
        Exmaine applications of proof-carrying code in WebAssembly.
    \item
        Analyze security policies and identify the guarantees they provide to a run-time system.
    \item
        Evaluate advantages and drawbacks of applying proof-carrying code techniques to WebAssembly in different domains.
    \item
        Implement proof validator and proof-generating compiler for a small, functional language
\end{enumerate}

\end{document}
